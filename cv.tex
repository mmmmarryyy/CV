\documentclass[11pt,a4paper]{moderncv} 


\usepackage[T2A]{fontenc} 
\usepackage[utf8]{inputenc} 
% possible options include font size ('10pt', '11pt' and '12pt'), paper size ('a4paper', 'letterpaper', 'a5paper', 'legalpaper', 'executivepaper' and 'landscape') and font family ('sans' and 'roman')

% moderncv themes
\moderncvstyle{classic}                        % style options are 'casual' (default), 'classic', 'oldstyle' and 'banking'
\moderncvcolor{blue}                          % color options 'blue' (default), 'orange', 'green', 'red', 'purple', 'grey' and 'black'
%\renewcommand{\familydefault}{\rmdefault}    % to set the default font; use '\sfdefault' for the default sans serif font, '\rmdefault' for the default roman one, or any tex font name
\nopagenumbers{}                             % uncomment to suppress automatic page numbering for CVs longer than one page


% adjust the page margins
\usepackage[scale=0.88]{geometry}
%\setlength{\hintscolumnwidth}{3cm}						% if you want to change the width of the column with the dates
%\AtBeginDocument{\setlength{\maketitlenamewidth}{6cm}}  % only for the classic theme, if you want to change the width of your name placeholder (to leave more space for your address details
\AtBeginDocument{\recomputelengths}                     % required when changes are made to page layout lengths

% personal data
\firstname{Maria}
\familyname{Barkovskaya}
\title{Curriculum Vitae}
\email{barkovskaya.maria@mail.ru}
\social[github]{https://github.com/mmmmarryyy}
\extrainfo{\textit{some names in CV are referal to github projects}}


%\nopagenumbers{}                             % uncomment to suppress automatic page numbering for CVs longer than one page


%----------------------------------------------------------------------------------
%            content
%----------------------------------------------------------------------------------
\begin{document}
\maketitle
\vspace{-12mm}
\section{Education}
\cventry{2021--nowadays}{The Faculty of Mathematics and Computer Science, Modern Software Engineering}{Saint-Petersburg State University}{Saint-Petersburg, Russia}
{a second-year student now}{}

\section{Software Development Experience}

\cventry{03.2022--nowadays}{Mobile Build Engineering}{assembly of a chat client based on the Matrix - Element protocol, with some of its own settings, but especially with a stitched company domain name, for iOS, Android, Mac&PC}{\textit{\textbf{Swift, Kotlin}}}{}{}
\cventry{12.2022}{\href{https://github.com/mmmmarryyy/ISA}{ISA}}{a translator program (disassembler), with which you can convert machine code into program text in assembly language; Familiarity with the RISC-V instruction set architecture}{\textit{\textbf{C++}}}{}{}
\cventry{11.2022--12.2022}{\href{https://github.com/mmmmarryyy/processor_cache_memory_simulator}{Processor-Cache-Memory Simulator}}{building a cache and modeling the processor-cache-memory system in the Verilog description language.}{\textit{\textbf{Verilog}}}{}{}
\cventry{11.2021--12.2021}{\href{https://github.com/mmmmarryyy/sport-management-system}{Sport Management System}}{a system of organization for sports competitions in one of the cyclic sports: running, cross-country skiing, swimming, cycling, orienteering, etc. (work in team of 3 people)}{\textit{\textbf{Kotlin}}}{}{}
\cventry{12.2021}{\href{https://github.com/mmmmarryyy/BMP-editor}{BMP Editor}}{BMP format image editor for cropping and rotating images}{\textit{\textbf{C}}}{}{}
\cventry{10.2021--11.2021}{Data Visualization}{Сonsole App for drawing common types of diagrams (pie charts, histograms, scatterplots, etc.), the program draws a diagram in a graphical window and saves the result to a file}{\textit{\textbf{Kotlin}}}{}{}
\cventry{10.2021}{\href{https://github.com/mmmmarryyy/KV-DataBase}{Key-Value DataBase}}{A console interface to the database that supports operations for searching, inserting and deleting text values by their corresponding text keys}{\textit{\textbf{Kotlin}}}{}{}
\cventry{09.2021}{Diff utility}{Сonsole app that compares two texts line by line and shows which lines need to be deleted/added in order to get one text from another one}{\textit{\textbf{Kotlin}}}{}{}


\section{Technical Strengths}
\cvlistitem{Advanced Git skills}
\cvlistitem{Knowledge of Basic Algorithms and Data Structures}
\cvlistitem{Basic Knowledge of OOP}


\section{Soft skills}
\cvlistitem{Team player. Able to easily find a common language with others}
\cvlistitem{Excellent time management skills. Сontrol my time and observe work-life balance}
\cvlistitem{Сonstant self-development. Сonstantly working on both technical skills and other areas of my life (hobbies, relationships, financial literacy, etc.)}

\section{Languages}
\cvline{Russian}{\small Native speaker}
\cvline{English}{\small B1-B2 level}

\end{document}

%% end of file `template.tex'.